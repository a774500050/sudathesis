\chapter{红楼梦}
    红楼梦的微分形式。
    \begin{itemize}
        \item Gauss's law
        \begin{equation}
            \nabla\cdot\mathbf{D}=\rho
        \end{equation}
        \item Guass's law for magnetism
        \begin{equation}
            \nabla\cdot\mathbf{H}=0
        \end{equation}
        \item Maxwell-Faraday's law
        \begin{equation}
            \nabla \times \mathbf{E}=-\frac{\partial \mathbf{B}}{\partial t}
        \end{equation}
        \item Maxwell-Ampere's law
        \begin{equation}
            \nabla \times \mathbf{H}=\mathbf{\delta}+\frac{\partial \mathbf{D}}{\partial t}
        \end{equation}
    \end{itemize}
    \section{第一回\hspace{0.5em}甄士隐梦幻识通灵\hspace{0.5em}贾雨村风尘怀闺秀}
        此开卷第一回也。作者自云:曾历过一番梦幻之后,故将真事隐去,而借通灵说此《石头记》一书也,故曰“甄士隐”云云。但书中所记何事何人?自己又云:“今风尘碌碌,一事无成,忽念及当日所有之女子,一一细考较去,觉其行止见识皆出我之上。我堂堂须眉诚不若彼裙钗,我实愧则有馀,悔又无益,大无可如何之日也。当此日,欲将已往所赖天恩祖德,锦衣纨裦之时,饫甘餍肥之日,背父兄教育之恩,负师友规训之德,以致今日一技无成、半生潦倒之罪,编述一集,以告天下;知我之负罪固多,然闺阁中历历有人,万不可因我之不肖,自护己短,一并使其泯灭也。所以蓬牖茅椽,绳床瓦灶,并不足妨我襟怀;况那晨风夕月,阶柳庭花,更觉得润人笔墨。我虽不学无文,又何妨用假语村言敷演出来?亦可使闺阁昭传。复可破一时之闷,醒同人之目,不亦宜乎?”故曰“贾雨村”云云。更于篇中间用“梦”“幻”等字,却是此书本旨,兼寓提醒阅者之意。

        看官你道此书从何而起?说来虽近荒唐,细玩颇有趣味。却说那女娲氏炼石补天之时,于大荒山无稽崖炼成高十二丈、见方二十四丈大的顽石三万六千五百零一块。那娲皇只用了三万六千五百块,单单剩下一块未用,弃在青埂峰下。谁知此石自经锻炼之后,灵性已通,自去自来,可大可小。因见众石俱得补天,独自己无才不得入选,遂自怨自愧,日夜悲哀。一日正当嗟悼之际,俄见一僧一道远远而来,生得骨格不凡,丰神迥异,来到这青埂峰下,席地坐谈。见着这块鲜莹明洁的石头,且又缩成扇坠一般,甚属可爱。那僧托于掌上,笑道:“形体倒也是个灵物了,只是没有实在的好处。须得再镌上几个字,使人人见了便知你是件奇物,然后携你到那昌明隆盛之邦、诗礼簪缨之族、花柳繁华地、温柔富贵乡那里去走一遭。”石头听了大喜,因问:“不知可镌何字?携到何方?望乞明示。”那僧笑道:“你且莫问,日后自然明白。”说毕,便袖了,同那道人飘然而去,竟不知投向何方。
        \subsection{齐物论}
            非彼无我,非我无所取。是亦近矣,而不知其所为使。若有真宰,而特不得其眹。可行己信,而不见其形,有情而无形。百骸、九窍、六藏,赅而存焉,吾谁与为亲?汝皆说之乎?其有私焉?如是皆有为臣妾乎?其臣妾不足以相治乎?其递相为君臣乎?其有真君存焉?如求得其情与不得,无益损乎其真。一受其成形,不忘以待尽。与物相刃相靡,其行尽如驰,而莫之能止,不亦悲乎!终身役役而不见其成功,苶然疲役而不知其所归,可不哀邪!人谓之不死,奚益!其形化,其心与之然,可不谓大哀乎?人之生也,固若是芒乎?其我独芒,而人亦有不芒者乎?夫随其成心而师之,谁独且无师乎?奚必知代而心自取者有之?愚者与有焉。未成乎心而有是非,是今日适越而昔至也。是以无有为有。无有为有,虽有神禹,且不能知,吾独且奈何哉!

            夫言非吹也,言者有言,其所言者特未定也。果有言邪?其未尝有言邪?其以为异于鷇音,亦有辩乎,其无辩乎?道恶乎隐而有真伪?言恶乎隐而有是非?道恶乎往而不存?言恶乎存而不可?道隐于小成,言隐于荣华。故有儒墨之是非,以是其所非而非其所是。欲是其所非而非其所是,则莫若以明。

            物无非彼,物无非是。自彼则不见,自知则知之。故曰彼出于是,是亦因彼。彼是方生之说也,虽然,方生方死,方死方生;方可方不可,方不可方可;因是因非,因非因是。是以圣人不由,而照之于天,亦因是也。是亦彼也,彼亦是也。彼亦一是非,此亦一是非。果且有彼是乎哉?果且无彼是乎哉?彼是莫得其偶,谓之道枢。枢始得其环中,以应无穷。是亦一无穷,非亦一无穷也。故曰莫若以明以指喻指之非指,不若以非指喻指之非指也;以马喻马之非马,不若以非马喻马之非马也。天地一指也,万物一马也。
