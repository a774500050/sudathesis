% 结论
\begin{conclusion}
《战国策》又名《国策》,主要记载战国时期谋臣策士纵横捭阖的斗争。全书按东周、西周、秦国、齐国、楚国、赵国、魏国、韩国、燕国、宋国、卫国、中山国依次分国编写,分为12策,共33卷,共497篇。 所记载的历史,上起前490年智伯灭范氏,下至前221年高渐离以筑击秦始皇,约12万字。是先秦历史散文成就最高,影响最大的著作之一。

《战国策》是我国古代记载战国时期政治斗争的一部最完整的著作。它实际上是当时纵横家游说之辞的汇编,而当时七国的风云变幻,合纵连横,战争绵延,政权更迭,都与谋士献策、智士论辩有关,因而具有重要的史料价值。该书文辞优美,语言生动,富于雄辩与运筹的机智,描写人物绘声绘色,常用寓言阐述道理,著名的寓言就有“画蛇添足”“亡羊补牢”“狡兔三窟”“狐假虎威”等。在我国古典文学史上亦占有重要地位。
\end{conclusion}

% 致谢
\begin{acknowledgements}
赵惠文王三十年,相都平君田单问赵奢曰:“吾非不说将军之兵法也,所以不服者独将军之用众。用众者,使民不得耕作,粮食輓赁不可给也。此坐而自破之道也,非单之所为也。单闻之,帝王之兵所用者不过三万而天下服矣。今将军必负十万、二十万之众乃用之,此单之所不服也。”

马服曰:“君非徒不达于兵也,又不明其时势。夫吴干之剑,肉试则断牛马,金试则截盘匜;薄之柱上而击之则折为三,质之石上而击之则碎为百。今以三万之众而应强国之兵,是薄柱击石之类也。且夫吴干之剑材难,夫毋脊之厚而锋不入,无脾之薄而刃不断兼有是两者,无钓、罕、镡蒙须之便,操其刃而刺,则未入而手断。君无十馀、二十万之众,而为此钓、罕、镡蒙须之便,而徒以三万行于天下,君焉能乎?且古者四海之内分为万国,城虽大无过三百丈者,人虽众无过三千家者,而以集兵三万距,此奚难哉!今取古之为万国者分以为战国七,能具数十万之兵,旷日持久数岁,即君之齐已。齐以二十万之众攻荆,五年乃罢;赵以二十万之众攻中山,五年乃归。今者齐、韩卫相方而国围攻焉,岂有敢曰我其以三万救是者乎哉?今千丈之城、万家之邑相望也,而索以三万之众围千丈之城,不存其一角,而野战不足用也,君将以此何之?”都平君喟然太息曰:“单不至也!”
\end{acknowledgements}
