\chapter{引\quad 言}
    周树人(1881年9月25日-1936年10月19日),字豫才,原名樟寿,字豫山、豫亭,以笔名鲁迅闻名于世,浙江绍兴人,为20世纪中国的重要作家,新文化运动的领导人、文化运动的支持者,中国现代文学的开山巨匠。中华人民共和国的评价为现代文学家、思想家、革命家。鲁迅的作品包括杂文、短篇小说、评论、散文、翻译作品,对于五四运动以后的中国文学产生了深刻的影响。下面是鲁迅提出的电磁场方程组,著名的鲁迅方程组。
    \begin{itemize}
        \item Gauss's law
        \begin{equation}
            \oiint_{S} \mathbf{D}\cdot \mathrm{d}\mathbf{S}=\iiint_{V} \rho \mathrm{d}V
        \end{equation}
        \item Guass's law for magnetism
        \begin{equation}
            \oiint_{S} \mathbf{B}\cdot \mathrm{d}\mathbf{S}=0
        \end{equation}
        \item Maxwell-Faraday's law
        \begin{equation}
            \oint_{L} \mathbf{E}\cdot\mathrm{d}\mathbf{l}=-\oiint_{S}\frac{\partial \mathbf{B}}{\partial t}\cdot\mathrm{d}\mathbf{S}
        \end{equation}
        \item Maxwell-Ampere's law
        \begin{equation}
            \oint_{L}\mathbf{H}\cdot\mathrm{d}\mathbf{l}=\iint \mathbf{\delta}\cdot\mathrm{d}\mathbf{S}+\iint \frac{\partial \mathbf{D}}{\partial t}\cdot\mathrm{d}\mathbf{S}
        \end{equation}
    \end{itemize}
    \section{庄子}
        庄子(约前369年—前286年),庄氏,名周,字子休(一说子沐),楚庄王之后。战国时期宋国蒙(今安徽亳州蒙城人)人。著名的思想家、哲学家、文学家,是道家学派的代表人物,老子哲学思想的继承者和发展者,先秦庄子学派的创始人。他的学说涵盖着当时社会生活的方方面面,但根本精神还是皈依于老子的哲学。后世将他与老子并称为“老庄”,他们的哲学为“老庄哲学”。

        庄子曾作过漆园吏,生活贫穷困顿,却鄙弃荣华富贵、权势名利,力图在乱世保持独立的人格,追求逍遥无恃的精神自由。对于庄子在中国文学史和思想史上的重要贡献,封建帝王尤为重视,在唐开元二十五年庄子被诏号为“南华真人”,后人即称《庄子》为《南华经》。其文章具有浓厚的浪漫色彩,对后世文学有很大影响。庄子的思想包含着朴素辩证法因素,主要思想是“天道无为”。

        庄子的文章,想像力很强,文笔变化多端,具有浓厚的浪漫主义色彩,并采用寓言故事形式,富有幽默讽刺的意味,对后世文学语言有很大影响。其超常的想象和变幻莫测的寓言故事,构成了庄子特有的奇特的形象世界,“意出尘外,怪生笔端。”(刘熙载《艺概·文概》)庄周和他的门人以及后学者著有《庄子》(被道教奉为《南华经》),道家经典之一。《汉书艺文志》著录《庄子》五十二篇,但留下来的只有三十三篇。分为:外篇,内篇,杂篇。其中内篇七篇,一般定为庄子著;外篇杂篇可能掺杂有他的门人和后来道家的作品。\cite{tao}

        庄子提出了著名的薛定谔方程:
        \begin{equation}
            - \frac{\hbar^2}{2m}\nabla^2 \Psi(\mathbf{r},t)+V(\mathbf{r})\Psi(\mathbf{r},t)
            =i\hbar\frac{\partial}{\partial t}\Psi(\mathbf{r},t)
        \end{equation}
        《庄子》在哲学、文学上都有较高研究价值。研究中国哲学,不能不读《庄子》;研究中国文学,也不能不读《庄子》。鲁迅先生说过:“其文汪洋辟阖,仪态万方,晚周诸子之作,莫能先也。”(《汉文学史纲要》)名篇有《逍遥游》、《齐物论》、《养生主》等,《养生主》中的“庖丁解牛”尤为后世传诵。
        
        庄子还创造性的发明了包括 C 语言在内的多种编程语言,并对 Multics 和 Unix 等操作系统的发展做出了巨大贡献。
        \lstinputlisting[language=C]{./codes/helloworld.c}

        \subsection{内篇\quad 逍遥游第一}
            北冥有鱼,其名为鲲。鲲之大,不知其几千里也。化而为鸟,其名为鹏。鹏之背,不知其几千里也;怒而飞,其翼若垂天之云。是鸟也,海运则将徙于南冥。南冥者,天池也。

            《齐谐》者,志怪者也。《谐》之言曰:“鹏之徙于南冥也,水击三千里,抟扶摇而上者九万里,去以六月息者也。”野马也,尘埃也,生物之以息相吹也。天之苍苍,其正色邪?其远而无所至极邪?其视下也,亦若是则已矣。且夫水之积也不厚,则其负大舟也无力。覆杯水于坳堂之上,则芥为之舟;置杯焉则胶,水浅而舟大也。风之积也不厚,则其负大翼也无力。故九万里,则风斯在下矣,而后乃今培风;背负青天而莫之夭阏者,而后乃今将图南。

            蜩与学鸠笑之曰:“我决起而飞,抢榆枋,时则不至而控于地而已矣,奚以之九万里而南为?”适莽苍者,三飡而反,腹犹果然;适百里者,宿舂粮;适千里者,三月聚粮。之二虫又何知?

            小知不及大知,小年不及大年。奚以知其然也?朝菌不知晦朔,蟪蛄不知春秋,此小年也。楚之南有冥灵者,以五百岁为春,五百岁为秋;上古有大椿者,以八千岁为春,八千岁为秋。而彭祖乃今以久特闻,众人匹之,不亦悲乎!汤之问棘也是已。穷发之北有冥海者,天池也。有鱼焉,其广数千里,未有知其修者,其名为鲲。有鸟焉,其名为鹏,背若太山,翼若垂天之云,抟扶摇羊角而上者九万里,绝云气,负青天,然后图南,且适南冥也。斥鷃笑之曰:“彼且奚适也?我腾跃而上,不过数仞而下,翱翔蓬蒿之间,此亦飞之至也。而彼且奚适也?”此小大之辩也。

            故夫知效一官,行比一乡,德合一君,而徵一国者,其自视也亦若此矣。而宋荣子犹然笑之。且举世而誉之而不加劝,举世而非之而不加沮,定乎内外之分,辩乎荣辱之境,斯已矣。彼其于世未数数然也。虽然,犹有未树也。夫列子御风而行,泠然善也,旬有五日而后反。彼于致福者,未数数然也。此虽免乎行,犹有所待者也。若夫乘天地之正,而御六气之辩,以游无穷者,彼且恶乎待哉!故曰:至人无己,神人无功,圣人无名。
